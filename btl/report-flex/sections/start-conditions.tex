\documentclass[../report.tex]{subfiles}

\begin{document}
Flex cung cấp cơ chế cho việc kích hoạt các luật dưới những điều kiện cụ thể. 
Bất kì một luật nào được bắt đầu bằng \textbf{'<sc>'} sẽ chỉ được 
kích hoạt khi trình phân tích từ tố (scanner) ở trong 
một "điều kiện bắt đầu" cụ thể, ở đây là \textbf{sc}. 
Điều kiện bắt đầu
có thể hiểu đơn giản là những 
trạng thái mà scanner có thể chuyển tới, 
những trạng thái này quyết định những luật (rules) nào được kích hoạt 
tại thời điểm đó.
Ví dụ: 
\begin{lstlisting}
<STRING>[a-zA-Z]*               do_this();
<INITIAL,STRING,QUOTE>"abc"     do_that();
\end{lstlisting}
Trong đó 
\textbf{STRING},
\textbf{INITIAL},
\textbf{QUOTE}
là các điều kiện bắt đầu. \\
\textbf{INITIAL} là điều kiện bắt đầu (hay trạng thái) 
ban đầu của scanner. Luật \textbf{[a-zA-Z]*} chỉ hoạt động khi 
điều kiện bắt đầu là \textbf{STRING}. Luật 
\textbf{"abc"} chỉ hoạt động khi điều kiện bắt đầu là một 
trong ba khả năng: 
\textbf{STRING},
\textbf{INITIAL} và
\textbf{QUOTE}. 
Nếu ta không chỉ định điều kiện bắt đầu thì mặc định sẽ là 
\textbf{INITIAL}:
\begin{lstlisting}
ABCD[0-9] do_somthing();
\end{lstlisting}
Tương đương với:
\begin{lstlisting}
<INITIAL>ABCD[0-9] do_somthing();
\end{lstlisting}

Để sử dụng được các điều kiện bắt đầu, trước tiên 
ta phải khai báo chúng, có hai cách khai báo: 
\begin{lstlisting}
%s EXAMPLE1
%x EXAMPLE2
\end{lstlisting}
Các khai báo thứ nhất: \textbf{'\%s'} 
được gọi là kiểu khai báo 'inclusive'. \\
Các khai báo thứ hai: \textbf{'\%x'} được gọi là kiểu 
khai báo 'exclusive'.

Điểm khác nhau giữa chúng là: Những luật không bắt đầu bởi điều kiện bắt đầu sẽ mặc định bao gồm
\textbf{INITIAL} và các điều kiện bắt đầu 'inclusive', ví dụ:
\begin{lstlisting}
"xyz"   something();
\end{lstlisting}
Tương đương với:
\begin{lstlisting}
<INITIAL, EXAMPLE1>"xyz"   something();
\end{lstlisting}
Hay những điều kiện bắt đầu 'exclusive' sẽ không được thêm vào 
danh sách mặc định. 

\noindent Ta có thể chuyển đổi điều kiện bắt đầu bằng cách gọi hàm:
\begin{lstlisting}
BEGIN(some_start_condition)
\end{lstlisting}
Trong các \textbf{action}.

Trong báo cáo này sẽ sử dụng các điều kiện bắt đầu để loại bỏ các 
comment (một dòng và nhiều dòng) trong ngôn ngữ PL/0 mở rộng. 
Ví dụ cho comment một dòng:
\begin{lstlisting}
%x COMMENT
%%
"//"            BEGIN(COMMENT);
<COMMENT>"\n"   BEGIN(INITIAL);
<COMMENT>.      ;
%%
...
\end{lstlisting}

\end{document}
