\documentclass[../report.tex]{subfiles}

\begin{document}
\cite{c-interface} Chương trình được Bison sinh ra là một chương 
trình C và hàm thực sự thực hiện quá trình phân tích cú 
pháp là \textbf{yyparse}.
\subsection{Hàm phân tích cú pháp yyparse} 
Để quá trình phân tích cú pháp được thực thi, ta gọi 
hàm \textbf{yyparse}. Hàm này sẽ đọc các token, 
thực thi các \textbf{action} trong phần 
Grammar Rules tương ứng, rồi cuối cùng trả về một giá trị 
khi gặp kí tự báo kết thúc file hoặc khi không thể phục hồi 
từ lỗi. Ta cũng có thể chỉ định cho hàm \textbf{yyparse}
trả về ngay lập tức mà không đọc thêm từ tố nào nữa. 

\begin{lstlisting}
int yyparse(void);
\end{lstlisting}
Hàm trả về $0$ nếu như quá trình phân tích cú pháp thành công (hàm 
trả về do gặp kí tự kết thúc file). \\
Hàm trả về $1$ nếu như gặp lỗi do đầu vào không hợp lệ (Ví dụ: Chuỗi
token đầu vào không thỏa mãn cú pháp đã được chỉ định) hoặc do 
hàm \textbf{YYABORT} được gọi.  \\
Hàm trả về $2$ nếu như trình phân tích sử dụng hết bộ nhớ cho phép. 

Trong một \textbf{action} bất kì của phần Grammar Rules, ta có 
thể chỉ định cho hàm \textbf{yyparse} trả về ngay lập tức bằng 
cách sử dụng 2 macro sau: 
\begin{itemize}
\item \textbf{YYACCEPT}: Trả về ngay lập tức với giá trị 0 
(thành công)
\item \textbf{YYABORT}: Trả về ngay lập tức với giá trị 1 
(lỗi)
\end{itemize}

\subsection{Hàm phân tích từ tố yylex}
Hàm \textbf{yylex} là hàm nhận diện các từ tố từ một xâu đầu vào 
và trả về các giá trị từ tố đó cho parser sử dụng. 
Bison không tạo hàm này một cách từ động mà ta cần phải tạo nó 
để hàm \textbf{yyparse} có thể gọi nó. 

Trong một chương trình đơn giản, ta có thể tạo hàm \textbf{yylex}
ngay trong phần Epilogue. Nhưng nếu \textbf{yylex} được 
định nghĩa ở một file riêng (ví dụ như dùng Flex để sinh) 
thì ta cần phải cho phép file đó có thể đọc được định nghĩa 
của các token (do sử dụng \textbf{\%token}). 
Để làm điều đó, ta sử dụng tùy chọn '-d' khi chạy Bison, để cho 
Bison sẽ ghi những định nghĩa token và các định nghĩa macro cần 
thiết ra một file header riêng biệt, thông thường là 
\textbf{name.tab.h}.

Quy ước của hàm \textbf{yylex}:
\begin{lstlisting}
int yylex(void);
\end{lstlisting}
Hàm phải trả về một giá trị số không âm cho các token 
đã được nhận diện; giá trị 0 hoặc giá trị âm báo hiệu kết thúc file. 

Quy ước này được thiết kế để đầu ra của chương trình 
sinh bởi lex/flex có thể được sử dụng làm đầu vào cho Bison. 

\subsection{Hàm báo lỗi yyerror}
Khi parser sinh bởi Bison phát hiện một lỗi cú pháp bởi đọc một token 
mà không có sản xuất nào thỏa mãn. Ngoài ra, một \textbf{action}
trong Grammar Rules sử dụng macro \textbf{YYERROR} để báo lỗi. 

Khi parser sinh bởi Bison muốn báo lỗi, nó sử dụng hàm 
\textbf{yyerror} mà ta bắt buộc phải khai báo và định nghĩa. 
Hàm \textbf{yyerror} có một tham số đầu vào là một xâu, thông 
thường là xâu ``syntax error''.

Parser cũng có thể phát hiện một loại lỗi khác: cạn kiệt bộ nhớ. 
Nó có thể xảy ra khi mà chuỗi token đầu vào chứa một cấu trúc 
lồng nhau rất sâu. Nhưng lỗi này thường không xảy ra do 
parser tạo bởi Bison sẽ tự động mở rộng kích thước stack khi 
nó chạm tới ngưỡng. Xâu được truyền vào hàm \textbf{yyerror}
là: ``memory exhausted''. 

Hàm \textbf{yyerror} được định nghĩa đơn giản như sau: 
\begin{lstlisting}
void yyerror(const char *s) {
    fprintf(stderr, ``%s\n'', s);
}
\end{lstlisting}
Tuy nhiên ta cần khai báo hàm này trong phần Prologue để tránh 
trình dịch báo lỗi. 

Sau khi hàm \textbf{yyerror} trả về, hàm \textbf{yyparse} 
sẽ cố gắng khôi phục lỗi nếu ta đã viết những Grammar Rule phù 
hợp.


\end{document}
