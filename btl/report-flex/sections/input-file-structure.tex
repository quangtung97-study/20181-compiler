\documentclass[../report.tex]{subfiles}

\begin{document}
Mỗi file chỉ dẫn đầu vào cho Flex bao gồm 3 vùng, được phân tách nhau 
bởi các dòng chỉ chứa xâu '\%\%': \cite{flex-manual} \\
\lstinputlisting[style=customc]{src/input-file-format.l}

\noindent Trong đó: 
\begin{itemize}
\item \textbf{definitions}: Vùng chứa các định nghĩa. 
\item \textbf{rules}: Vùng chứa luật của các từ tố. 
\item \textbf{user code}: Vùng chứa mã C/C++ được thêm vào.
\end{itemize}

\subsection{Vùng định nghĩa - Definitions Section}
Vùng định nghĩa chứa các định nghĩa tên và các định nghĩa điều kiện bắt đầu.
Định nghĩa tên có dạng:
\begin{lstlisting}
name definition
\end{lstlisting}
\textbf{'name'} là các tên có cú pháp giống như các tên trong ngôn ngữ C.
\textbf{'definition'} là phần được lấy từ kí tự không phải kí tự trắng đầu tiên 
sau \textbf{'name'} và đến hết dòng. 
\textbf{'name'} không được thụt lề đầu dòng. 
Các định nghĩa có thể được tham chiếu đến bằng cách sử dụng \textbf{'\{name\}'} 
và nó sẽ được chuyển thành \textbf{'(definition)'}. \\
Ví dụ: 
\begin{lstlisting}
DIGIT [0-9]
\end{lstlisting}
Thì \textbf{'\{DIGIT\}'} sẽ được chuyển thành \textbf{'([0-9])'}.

Bất kì một dòng nào trong vùng này được viết thụt lề đầu dòng hoặc 
được bao xung quanh bởi \textbf{\%\{} và \textbf{\%\}} thì sẽ được sao chép nguyên văn 
vào file mã nguồn sinh ra. \\
Ví dụ: 
\lstinputlisting[style=customc]{src/unindented-line.l}
Thì đoạn code sau sẽ được sao chép nguyên văn vào file đầu ra: 
\lstinputlisting[style=customc]{src/unindented-line-output.c}

\subsection{Vùng các luật - Rules Section}
Là vùng bao gồm một chuỗi các luật có dạng:
\begin{lstlisting}
pattern action
\end{lstlisting}
Trong đó, \textbf{pattern} không được phép thụt lề đầu dòng. 
Và phần \textbf{action} phải được bắt đầu trên cùng một dòng. 

Ở trong vùng các luật, bất kì một dòng nào mà có thụt lề đầu dòng
hoặc các dòng được bao bởi \textbf{\%\{ \%\}} nằm trước luật đầu tiên 
có thể được sử dụng để khai báo những biến địa phương cho 
hàm phân tích từ tố. Và các đoạn code này sẽ được thực thi mỗi 
khi hàm phân tích từ tố được gọi. 

Các \textbf{pattern} là các biểu thức chính quy phiên bản mở rộng, bao gồm:
\begin{itemize}
\item \textbf{'x'} \\
    Nhận diện được kí tự 'x'.
\item \textbf{'.'} \\
    Bất kì cứ kị tự nào ngoại trừ dấu xuống dòng. 
\item \textbf{'[xyz]'} \\
    Là một ``Lớp các kí tự''. Nhận diện được một trong các kí tự: 'x', 'y', 'z'.
\item \textbf{'[abj-oZ]'} \\
    Là một ``Lớp các kí tự''. Nhận diện được kí tự 'a', 'b', bất kì kí tự nào 
    trong khoảng từ 'j' đến 'o', hoặc kí tự 'Z'.
\item \textbf{'[\textasciicircum A-Z\textbackslash n]'} \\
    Là một ``Lớp phủ định các kí tự''. Nhận diện được bất kì kí tự nào không nằm trong khoảng 
    được chỉ định. Ở đây, khoảng đó là các kí tự từ 'A' tới 'Z' và kí tự xuống dòng. 
\item \textbf{'r*'} \\
    Không hay nhiều \textbf{r}. Trong đó \textbf{r} là một biểu thức chính quy bất kì. 
\item \textbf{'r+'} \\
    Một hay nhiều \textbf{r}. Trong đó \textbf{r} là một biểu thức chính quy bất kì. 
\item \textbf{'r?'} \\
    Không hay một \textbf{r}. Trong đó \textbf{r} là một biểu thức chính quy bất kì. 
\item \textbf{'r\{2, 5\}'} \\
    \textbf{r} được xuất hiện liên tiếp từ 2 đến 5 lần. Trong đó \textbf{r} là một biểu thức chính quy bất kì. 
\item \textbf{'r\{3.\}'} \\
    \textbf{r} được xuất hiện từ 3 lần trở lên. Trong đó \textbf{r} là một biểu thức chính quy bất kì. 
\item \textbf{'r\{4\}'} \\
    \textbf{r} được xuất hiện chính xác 4 lần. Trong đó \textbf{r} là một biểu thức chính quy bất kì. 
\item \textbf{'\{name\}'} \\
    Thay thế định nghĩa 'name' thành 'definition' đã được chỉ định ở vùng định nghĩa. 
\item \textbf{'rs'} \\
    Phép nối biểu thức chính quy. 
\item \textbf{'r|s'} \\
    Phép hợp biểu thức chính quy. 
\item \textbf{'\textasciicircum r'} \\
    Nhận diện các xâu từ \textbf{r} nhưng phải bắt đầu dòng. 
\item \textbf{'r\$'} \\
    Nhận diện các xâu từ \textbf{r} nhưng phải kết thúc dòng. 
    
\end{itemize}

\subsection{Vùng mã nguồn bổ sung - User Code Section}
Các dòng code trong vùng này đơn giản được sao chép vào file mã nguồn đầu ra. 
Vùng này có thể không có, có thể bỏ '\%\%' cuối cùng nếu vùng này trống rỗng.  

\end{document}
