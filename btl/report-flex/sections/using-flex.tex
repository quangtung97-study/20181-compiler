\documentclass[../report.tex]{subfiles}

\begin{document}
\subsection{Hoạt động của Flex}

\subsection{Ví dụ một file đầu vào Flex đơn giản} 

\subsection{Sử dụng Flex bằng dòng lệnh} 
Một file chỉ thị đầu vào cho Flex có đuôi thông thường là \textbf{.l}, ví dụ \textbf{scanner.l}. 
Để biên dịch file này sang file C/C++, ta sử dụng:
\begin{verbatim}
$ flex scanner.l
\end{verbatim}
Mặc định lệnh trên sẽ sinh ra một file \textbf{lex.yy.c}.
Ta có thể dịch file này bằng một trình dịch C thông thường, ví dụ \textbf{gcc}:
\begin{verbatim}
$ gcc lex.yy.c -o main
\end{verbatim}
Ta có thể chỉ định tên file xuất ra bởi flex bằng cách sử dụng tùy chọn \textbf{-o <filename>}:
\begin{verbatim}
$ flex -o scanner.lex.cpp scanner.l
\end{verbatim}

\end{document}
