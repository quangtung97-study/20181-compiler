\documentclass[../report.tex]{subfiles}

\begin{document}

\subsection{Bộ kí tự đầu vào} 
Bao gồm các kí tự sau: 
\begin{itemize}
\item Chữ cái: A..Z, a..z
\item Chữ số: 0..9
\item Kí tự đặc biệt: ' " , : ? ; > < = . \textbackslash \textasciicircum 
            + - * / \% \_ \& | \textasciitilde ! ( ) [ ] \{ \}
\end{itemize}

\subsection{Các từ vựng của Mini-C}
Bao gồm:
\begin{itemize}
    \item Hằng số: Số nguyên, số thực hoặc kí tự (Mã ASCII).
    \item Xâu
    \item Định danh 
    \item Từ khóa: 
        break, case, char, const, continue, 
        default, do, double, else, enum, 
        float, for, goto, if, int, long, 
        return, short, sizeof, struct, switch, 
        union, void, unsigned, while
    \item Dấu phép toán:
    \begin{itemize}
        \item Hậu tố / tiền tố: . -> ++ --
        \item Số học: + - * / \% 
        \item Bit: \textasciitilde \& \textasciicircum | 
        \item Logic: ! \&\& || 
        \item Dịch bit: << >> 
        \item Đặc biệt: ? , 
    \end{itemize}
    \item Dấu ngoặc: ( ) [ ] \{ \}
    \item Kết thúc lệnh / nhãn: : ;
    \item Phép gán: 
        = += -= *= /= \%= \textasciicircum= |= <<= >>=
\end{itemize}
\subsection{Phân tích từ vựng bằng Flex \& Bison}
\subsubsection{Các từ tố được khái báo trong file Bison}
\begin{lstlisting}
%token KEYWORD
%token IDENT
%token CONSTANT
%token STRING_LITERAL

%token BREAK
%token CASE
%token CHAR
%token CONST
%token CONTINUE
%token DEFAULT
%token DO
%token DOUBLE
%token ELSE
%token ENUM
%token FLOAT
%token FOR
%token GOTO
%token IF
%token INT
%token LONG
%token RETURN
%token SHORT
%token SIZEOF
%token STRUCT
%token SWITCH
%token UNION
%token UNSIGNED
%token VOID
%token WHILE

%token ARROW
%token INCREASE
%token DECREASE
%token SHIFT_LEFT
%token SHIFT_RIGHT

%token GT
%token LT
%token GE
%token LE
%token EQ
%token NE

%token AND
%token OR

%token ADD_ASSIGN
%token SUB_ASSIGN
%token MUL_ASSIGN
%token DIV_ASSIGN
%token REM_ASSIGN
%token SL_ASSIGN
%token SR_ASSIGN
%token AND_ASSIGN
%token OR_ASSIGN
%token XOR_ASSIGN
\end{lstlisting}

\subsubsection{Comment}
Khai báo các điều kiện bắt đầu: 
\begin{lstlisting}
%x COMMENT_SINGLE
%x COMMENT_MULTIPLE
\end{lstlisting}
Xử lý comment của Mini-C:
\begin{lstlisting}
<INITIAL>"//"           BEGIN(COMMENT_SINGLE);
<INITIAL>"/*"           BEGIN(COMMENT_MULTIPLE);
<COMMENT_SINGLE,COMMENT_MULTIPLE>.  ;
<COMMENT_SINGLE>"\n"    BEGIN(INITIAL);
<COMMENT_MULTIPLE>"\n"  ;
<COMMENT_MULTIPLE>"*/"  BEGIN(INITIAL);
\end{lstlisting}


\subsubsection{Hằng số}
Các định nghĩa được sử dụng: 
\begin{lstlisting}
DIGIT       [0-9]
HEXDIGIT    [0-9a-fA-F]
OCTDIGIT    [0-7]
EXP_PART    [eE][+-]?{DIGIT}+
EXP_PART_OPT    ({EXP_PART})?
C_CHAR      ([^'\\\n]|{ESCAPE_CHAR})
ESCAPE_CHAR \\['"?\\abfnrtv]
\end{lstlisting}
Xử lý hằng số: 
\begin{lstlisting}
[1-9]{DIGIT}* return CONSTANT;
0{OCTDIGIT}* return CONSTANT;
0x{HEXDIGIT}+ return CONSTANT;
0X{HEXDIGIT}+ return CONSTANT;

{DIGIT}*"."{DIGIT}+{EXP_PART_OPT} return CONSTANT;
{DIGIT}+"."{EXP_PART_OPT} return CONSTANT;
{DIGIT}+{EXP_PART} return CONSTANT;

"'"{C_CHAR}"'" return CONSTANT;
\end{lstlisting}
Các hằng số bao gồm số nguyên thập phân, số nguyên bát phân, 
số nguyên hexa, số phẩy động có phần mũ và kí tự (gồm các kí 
tự thông thường và các kí tự đặc biệt).
\subsubsection{Xâu}
Định nghĩa được sử dụng: 
\begin{lstlisting}
S_CHAR      ([^"\\\n]|{ESCAPE_CHAR})
\end{lstlisting}
Xử lý xâu:
\begin{lstlisting}
\"{S_CHAR}+\" return STRING_LITERAL;
\end{lstlisting}

\subsubsection{Các từ khóa}
\begin{lstlisting}
"break" return BREAK;
"case" return CASE;
"char" return CHAR;
"const" return CONST;
"continue" return CONTINUE;
"default" return DEFAULT;
"do" return DO;
"double" return DOUBLE;
"else" return ELSE;
"enum" return ENUM;
"float" return FLOAT;
"for" return FOR;
"goto" return GOTO;
"if" return IF;
"int" return INT;
"long" return LONG;
"return" return RETURN;
"short" return SHORT;
"sizeof" return SIZEOF;
"struct" return STRUCT;
"switch" return SWITCH;
"union" return UNION;
"void" return VOID;
"unsigned" return UNSIGNED;
"while" return WHILE;
\end{lstlisting}

\subsubsection{Định danh}
Các định nghĩa được sử dụng: 
\begin{lstlisting}
ALPHA       [_A-Za-z]
ALPHADIGIT  [_A-Za-z0-9]
\end{lstlisting}
Xử lý định danh: 
\begin{lstlisting}
{ALPHA}{ALPHADIGIT}* return IDENT;
\end{lstlisting}

\subsubsection{Dấu phép toán}
\begin{lstlisting}
"." return '.';
"->" return ARROW;
"++" return INCREASE;
"--" return DECREASE;

"+" return '+';
"-" return '-';
"*" return '*';
"/" return '/';
"%" return '%';
"~" return '~';
"!" return '!';

"<<" return SHIFT_LEFT;
">>" return SHIFT_RIGHT;

">" return GT;
"<" return LT;
">=" return GE;
"<=" return LE;
"==" return EQ;
"!=" return NE;

"^" return '^';
"|" return '|';
"&" return '&';

"&&" return AND;
"||" return OR;

"?" return '?';
"," return ',';
\end{lstlisting}

\subsubsection{Dấu ngoặc}
\begin{lstlisting}
"[" return '[';
"]" return ']';
"{" return '{';
"}" return '}';
"(" return '(';
")" return ')';
\end{lstlisting}

\subsubsection{Dấu kết thúc lệnh / nhãn}
\begin{lstlisting}
":" return ':';
";" return ';';
\end{lstlisting}

\subsubsection{Dấu phép gán}
\begin{lstlisting}
"=" return '=';
"+=" return ADD_ASSIGN;
"-=" return SUB_ASSIGN;
"*=" return MUL_ASSIGN;
"/=" return DIV_ASSIGN;
"%=" return REM_ASSIGN;
"&=" return AND_ASSIGN;
"|=" return OR_ASSIGN;
"^=" return XOR_ASSIGN;
"<<=" return SL_ASSIGN;
">>=" return SR_ASSIGN;
\end{lstlisting}

\subsubsection{Xử lý lỗi}
\begin{lstlisting}
{WHITESPACE}            ;
. { 
    printf("Khong nhan dien duoc ki tu: %s\n", yytext);
    exit(-1); 
}
\end{lstlisting}

\end{document}
