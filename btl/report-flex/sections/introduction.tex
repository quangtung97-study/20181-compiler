\documentclass[../report.tex]{subfiles}

\begin{document}
Flex là bộ sinh chương trình phân tích từ vựng, 
là một phần mềm miễn phí và mã nguồn mở thay thế cho Lex. \cite{flex-wiki}
Nó thường xuyên được sử dụng đi kèm với bộ sinh chương trình phân 
tích cú pháp Yacc (hoặc mã nguồn mở thay thế là GNU Bison) để tạo nên 2 khối 
chức năng cơ bản của một chương trình dịch. 

Flex được viết bằng ngôn ngữ C bởi Vern Paxson vào khoảng năm 1987. 
Flex nhận đầu vào là một file chỉ dẫn, thông thường có đuôi là \textbf{.l}.
Từ đó Flex có thể sinh ra mã C có thể biên dịch và thực thi mà 
không cần thêm bất kì thư viện ngoài nào. 

Mã C sinh ra bởi Flex sử dụng Automata hữu hạn đơn định (Deterministic Finite Automation - DFA) 
để thực hiện việc tách xâu đầu vào thành các từ tố tương ứng. 
Thuật toán thường có độ phức tạp thời gian tính là $O(n)$ với $n$ là độ dài 
xâu đầu vào. 

Flex chỉ có thể sinh ra mã code C hoặc C++. 

\end{document}
