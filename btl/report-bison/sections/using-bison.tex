\documentclass[../report.tex]{subfiles}

\begin{document}
\subsection{Lệnh sinh trình phân tích cú pháp}
Nếu ta có file grammar đầu vào là \textbf{example.y}, 
lệnh để sinh mã nguồn C từ file này là: 
\begin{verbatim}
$ bison -d -o example.yacc.c example.y
\end{verbatim}
Lệnh này sẽ tạo ra 2 file \textbf{example.yacc.c} và 
\textbf{example.yacc.h}. 
Tùy chọn \textbf{-d} nghĩa là sẽ tạo ra file header 
cùng với file mã nguồn. Tùy chọn \textbf{-o <filename>} 
chỉ định tên file sẽ được sinh ra. 

Nếu \textbf{-o example.yacc.cxx} hoặc \textbf{-o example.yacc.cpp}
thì file header tương 
ứng là \textbf{example.yacc.hxx} hoặc \textbf{example.yacc.hpp}

File \textbf{example.yacc.h} sẽ thường được include vào 
mã lex và mã nguồn chứa hàm main tương ứng. 

\subsection{Parser sinh bởi Bison đơn giản}
Ở đây ta dùng Bison để sinh chương trình tính toán 
biểu thức đơn giản.

File ngữ pháp bison: \textbf{calc.y}
\lstinputlisting[style=customc]{src/calc.y}

File chỉ thị của flex: \textbf{calc.l}
\lstinputlisting[style=customc]{src/calc.l}

File chứa hàm main: \textbf{main.c}
\lstinputlisting[style=customc]{src/main.c}

Để biên dịch chương trình: 
\begin{verbatim}
bison -d -o calc.yacc.c calc.y
flex -o calc.lex.c calc.l
gcc main.c calc.yacc.c calc.lex.c -o main
\end{verbatim}
 
Khi đó với file đầu vào: \textbf{input}
\begin{verbatim}
(-3 - 2) * (20 + 80) + 50
\end{verbatim}

Ta được đầu ra: 
\begin{verbatim}
$ ./main < input
-450
\end{verbatim}

\subsection{Biểu diễn ngữ pháp của PL0 trong Bison} 
\lstinputlisting[style=customc]{src/pl0.y}
Trong đó 
\begin{lstlisting}
% nonassoc THEN
% nonassoc ELSE
\end{lstlisting}
được sử dụng để giải quyết vấn đề nhập nhằng của 
văn phạm do ``Dangling else'' 
\cite{dangling-else} \cite{dangling-else-book}.

\end{document}
