\documentclass[../report.tex]{subfiles}

\begin{document}
\subsection{Các thành phần của một file văn phạm}
Một file mô tả văn phạm của Bison được chia thành 4 phần như sau: 
\cite{grammar-file}
\begin{lstlisting}
%{
    Prologue
%}

Bison declarations

%%
Grammar rules
%%

Epilogue
\end{lstlisting}
Trong đó, phần Prologue thường 
chứa các khai báo hàm, include của C. \\
Phần Bison Declarations chứa các khai báo được Bison sử dụng. \\
Phần Grammar Rules chứa các quy tắc ngữ pháp của ngôn ngữ đang cần biểu diễn và 
các lệnh thực hiện tương ứng. \\
Phần Epilogue thường chứa định nghĩa của các hàm C tương ứng với phần 
Prologue.

\subsection{Phần Prologue}
Phần Prologue chứa các định nghĩa macro và các khai báo của các 
hàm và biến được sử dụng trong các câu lệnh của phần Grammar Rules. 
\cite{prologue}

Những đoạn code trong phần này sẽ được sao chép vào phần đầu 
file mã nguồn của chương trình dịch được sinh ra. Có thể sử dụng 
\textbf{\#include ``\ldots''} để include các file header tương ứng. 

Có thể có nhiều hơn một phần Prologue nằm đan xen với các phần Bison Declarations.

\subsection{Phần Bison Declarations}
Phần Bison Declarations định nghĩa các kí hiệu được sử dụng 
để xây dựng ngữ pháp chương trình và kiểu dữ liệu của các giá trị. 
\cite{bison-declarations}

Tất cả token (ngoại trừ các token có một kí tự như '+' hoặc '*') phải được khai báo trước khi có thể sử dụng. 
Các kí hiệu không kết thúc phải được khai báo nếu muốn định nghĩa 
kiểu của kí hiệu đó để sử dụng làm giá trị ngữ nghĩa. 

Phần Bison Declarations cũng chỉ định kí hiệu không kết thúc bắt đầu
của văn phạm. Mặc định là sử dụng kí hiệu bên trái 
của rule đầu tiên trong phần Grammar Rules. 

Các khai báo trong phần Bison Declarations bao gồm: 
\begin{itemize}
\item \textbf{Khai báo require}:
Có thể chỉ định version nhỏ nhất của Bison trong khi xử lý 
file ngữ pháp. Nếu như yêu cầu về version không đủ, 
Bison dừng việc xử lý và báo lỗi. 
\begin{lstlisting}
%require ``version''
\end{lstlisting}

\item \textbf{Khai báo token}:
Để khai báo tên của token (kí tự kết thúc), ta sử dụng:
\cite{token-declarations}
\begin{lstlisting}
%token name
\end{lstlisting}
Bison sẽ chuyển khai báo token thành một hằng số trong file 
mã nguồn đầu ra (file header) để mà flex có thể sử dụng \textbf{name} để làm giá trị token cần trả về của hàm \textbf{yylex()}

Có thể chỉ định giá trị số của token bằng cú pháp: 
\begin{lstlisting}
%token NUM 300
\end{lstlisting}
Trong đó $300$ là giá trị số của token \textbf{NUM}

Nhưng tốt hơn cả là để Bison lựa chọn giá trị số cho tất cả các 
token. Bison sẽ tự động chọn các số mà không xung đột với giá trị của
các kí tự ASCII và các token trước đó. 
(Giá trị Bison chọn luôn lớn hơn 255). 

\item \textbf{Khai báo union và kiểu dữ liệu}:
Trong hầu hết các chương trình sẽ cần nhiều hơn một kiểu dữ liệu 
cho các token khác nhau. 
\cite{more-than-one} 
Như giữa token đại diện cho số và token đại diện cho xâu. 

Mặc định Bison chỉ có một kiểu giá trị ngữ nghĩa là int. 
Để sử dụng nhiều hơn một kiểu dữ liệu cho các giá trị ngữ
nghĩa, Bison yêu cầu trong file ngữ pháp phải có: 
\begin{itemize}
\item Chỉ định toàn bộ tập các kiểu dữ liệu sẽ được sử dụng (có thể
        sử dụng \textbf{\%union})
\item Trong các kí hiệu mà giá trị ngữ nghĩa được sử dụng 
    (kết thúc hoặc không kết thúc), 
    Chọn một trong các kiểu được chỉ định ở trên.
    (Sử dụng khai báo \textbf{\%token} 
        bổ sung kiểu cho các kí hiệu kết thúc,
    sử dụng \textbf{\%type} cho các kí hiệu không kết thúc)
\end{itemize}
Để khai báo union, ta dùng cú pháp: 
\begin{lstlisting}
%union {
    int num;
    char *name;
}
\end{lstlisting}
Để khai báo token có chứa kiểu, ta dùng cú pháp: 
\begin{lstlisting}
%token <num> NUMBER
\end{lstlisting}
Trong đó token \textbf{NUMBER} có kiểu là kiểu của 
biến \textbf{num} trong \textbf{union}.

Để khai báo kí hiệu không kết thúc có chứa kiểu, ta dùng: 
\begin{lstlisting}
%type <num> NONTERMINAL
\end{lstlisting}
Trong đó kí hiệu không kết thúc \textbf{NONTERMINAL} 
có kiểu là kiểu của biến \textbf{num} trong \textbf{union}. 
\cite{nonterminal-decl}

\item \textbf{Khai báo kí hiệu bắt đầu văn phạm}: 
\cite{start-symbol}
Bison mặc định rằng kí hiệu bắt đầu văn phạm là kí tự không
kết thúc đầu tiên bên trái trong Phần Grammar Rules. 
Có thể thay đổi kí hiệu này bằng: 
\begin{lstlisting}
%start symbol
\end{lstlisting}

\end{itemize}

Ngoài các phần kể trên, Bison còn hỗ trợ độ ưu tiên các toán tử
và thứ tự thực hiện của chúng (từ trái qua phải hoặc ngược lại). 
Tuy nhiên những điều đó có thể được biểu diễn bằng cách sử dụng văn
phạm phù hợp, dựa trên các kĩ thuật khử văn phạm nhập nhằng. 
Vì vậy trong báo cáo này sẽ không đề cập các phần đó. 

\subsection{Phần Grammar Rules}
Phần khai báo ngữ pháp của Bison bao gồm các grammar rule có 
dạng tổng quát như sau: 
\cite{syntax-rules}
\begin{lstlisting}
result: components...;
\end{lstlisting}
Trong đó \textbf{result} là kí hiệu không kết thúc 
mà rule này mô tả, và \textbf{components} là 
một chuỗi các kí hiệu kết thúc và không kết thúc tạo nên một 
sản xuất.

Ví dụ: 
\begin{lstlisting}
exp: exp '+' exp;
\end{lstlisting}
Mô tả một sản xuất $exp \rightarrow exp + exp$.
Các kí tự dấu trắng không ảnh hưởng đến định nghĩa các rule. 

Giữa các \textbf{component} có thể chứa các \textbf{action}
dùng để định nghĩa ngữ nghĩa của rule đó. 
Một \textbf{action} có dạng:
\begin{lstlisting}
{C statements}
\end{lstlisting}
\textbf{C statements} là một tập các câu lệnh của C được bao bọc 
trong cặp ngoặc nhọn, các lệnh này là các lệnh giống như được 
sử dụng bên trong định nghĩa hàm C. 
Bison không kiểm tra tính đúng đắn của các lệnh C này, chỉ copy 
nguyên vẹn vào file mã nguồn đầu ra. Các lệnh này sẽ được 
chạy khi mà đã xác định cây cú pháp và tiến hành thực hiện 
phân tích ngữ nghĩa. 

Nhiều các \textbf{rule} cho cùng một \textbf{result} có thể được 
định nghĩa bằng cách kết nối chúng bằng kí tự '|':
\begin{lstlisting}
result  :   rule1-components...
        |   rule2-components...
        ...
        ;
\end{lstlisting}
Ví dụ: 
\begin{lstlisting}
E   :   E '+' T     { printf(``E\n'');
    |   T           { printf(``T\n'');
    ;
\end{lstlisting}
Mô tả một sản xuất $E \rightarrow E + T | T$.

Một \textbf{rule} là trống rỗng nếu như bên phải của nó (các 
\textbf{component}) là trống rỗng. 
\cite{empty-rules}
Nó giống như kí hiệu $\varepsilon$ thể hiện một xâu rỗng. 

Ví dụ:
\begin{lstlisting}
E   :   TE1
E1  :   '+' T E1
    | 
    ;
\end{lstlisting}
Mô tả hai sản xuất là 
$E \rightarrow TE_1$
và 
$E_1 \rightarrow +TE_1 | \varepsilon$

Tuy nhiên việc không sử dụng kí tự gì khiến việc xác định rule 
trống rỗng khó khăn, ta có thể sử dụng \textbf{\%empty}:
\begin{lstlisting}
E   :   TE1
E1  :   '+' T E1
    |   %empty
    ;
\end{lstlisting}
Tuy nhiên \textbf{\%empty} là một mở rộng của Bison và 
không tồn tại trên Yacc.

Trong Bison ta nên khai báo các rule (hay các sản xuất) dưới dạng 
đệ quy trái thay vì đệ quy phải để tránh sử dụng stack trong quá trình hoạt động của parser. \cite{left-recursion}

\subsection{Phần Epilogue}
Những đoạn code trong phần Epilogue sẽ được sao chép nguyên vẹn 
vào cuối của file mã nguồn parser đẩu ra, giống như là code trong
phần Prologue được sao chép vào phần đầu của file đầu ra. 
Nếu như phần Epilogue trống rỗng, ta có thể bỏ đi \textbf{\%\%} 
kế thúc phần Grammar Rules.

\end{document}
