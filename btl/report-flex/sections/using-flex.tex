\documentclass[../report.tex]{subfiles}

\begin{document}
\subsection{Hoạt động của Flex}
Flex sẽ phân tích file chỉ dẫn đầu vào rồi tạo một DFA tương ứng. 
DFA này được mã hóa trong mã nguồn C/C++. DFA này sẽ ưu tiên những \textbf{pattern}
có độ dài càng lớn.

Chương trình sinh ra bởi Flex sẽ mặc định đọc dữ liệu từ \textbf{stdin}.
Ta có thể sử dụng hàm \textbf{void yyrestart(FILE *file)} để thiết lập 
con trỏ file mà chương sẽ sử dụng để đọc file đầu vào. 

Chương trình sẽ bắt đầu quá trình nhận diện xâu khi hàm \textbf{int yylex();} được gọi. 
Khi đó, nếu một xâu được nhận diện bởi một luận nào đó, mã nguồn C/C++ tương ứng với \textbf{pattern} được 
thực thi. Nếu đoạn mã đấy không có \textbf{return}, chương trình sẽ tiếp tục phân 
tích từ tố tiếp theo \cite{return-stmt}. Nếu có \textbf{return}, giá trị từ tố (hay giá trị \textbf{return})
được trả về qua giá trị trả về của hàm \textbf{yylex()}.
Xâu được nhận diện sẽ nằm trong biến toàn cục \textbf{yytext}, 
độ dài của xâu sẽ nằm trong biến toàn cục \textbf{yyleng}.

Nếu không một \textbf{pattern} nào nhận diện được xâu đầu vào, 
\textbf{luật mặc định} sẽ được thực thi: Kí tự tiếp theo trong file đầu vào sẽ 
được coi như là nhận diện được, và sẽ được in ra màn hình. 

Kích thước giới hạn của \textbf{yytext} được định nghĩa bởi hằng \textbf{YYLMAX}, 
là một số khá lớn. 
Ta có thể thay đổi nó đơn giản bởi định nghĩa lại: \textbf{\#define YYLMAX <number>}.

Hơn nữa, để có thể biên dịch được chương trình, ta cần phải định nghĩa hàm \textbf{yywrap}.
Hàm \textbf{int yywrap()} được gọi khi đầu vào đã được đọc hết. Nếu hàm trả về giá trị $1$, 
quá trình phân tích kết thúc, trả về 0 nếu như việc xử lý vẫn còn tiếp tục. Ta có thể 
thay đổi biến toàn cục \textbf{yyin} trỏ đến một file khác (hoặc sử dụng \textbf{yyrestart})
và chương trình sẽ tiếp tục phân tích từ file mới đó. \cite{yywrap-yyin}

Hoặc đơn giản ta có thể định nghĩa: 
\begin{lstlisting}
int yywrap() {
    return 1;
}
\end{lstlisting}
Trong vùng \textbf{User Code Section} của file chỉ thị.

\subsection{Ví dụ một file đầu vào Flex đơn giản} 

\subsection{Sử dụng Flex bằng dòng lệnh} 
Một file chỉ thị đầu vào cho Flex có đuôi thông thường là \textbf{.l}, ví dụ \textbf{scanner.l}. 
Để biên dịch file này sang file C/C++, ta sử dụng:
\begin{verbatim}
$ flex scanner.l
\end{verbatim}
Mặc định lệnh trên sẽ sinh ra một file \textbf{lex.yy.c}.
Ta có thể dịch file này bằng một trình dịch C thông thường, ví dụ \textbf{gcc}:
\begin{verbatim}
$ gcc lex.yy.c -o main
\end{verbatim}
Ta có thể chỉ định tên file xuất ra bởi flex bằng cách sử dụng tùy chọn \textbf{-o <filename>}:
\begin{verbatim}
$ flex -o scanner.lex.cpp scanner.l
\end{verbatim}

\end{document}
