\documentclass[../report.tex]{subfiles}

\begin{document}
Để điều khiển quá trình khôi phục lỗi trong Bison, ta sử dụng 
cách viết thêm các sản xuất mà trong đó chứa một token 
đặc biệt: \textbf{error}. Kí tự kết thúc (hay token) này luôn luôn
được định nghĩa và không thể được sử dụng cho các token thông thường 
khác. Parser sinh bởi Bison sẽ sinh ra một token \textbf{error}
khi mà gặp phải lỗi cú pháp. Nếu ta có token này trong sản xuất, 
quá trình phân tích cú pháp có thể được tiếp tục. 

Ví dụ: 
\begin{lstlisting}
stmts   :   %empty
        |   stmts '\n'
        |   stmts exp '\n'
        |   stmts error '\n'
\end{lstlisting}
Token \textbf{error} trong ví dụ này chỉ định rằng lỗi theo sau 
là một dấu xuống dòng là hợp lệ với sản xuất cuối cùng này. 

Vậy điều gì sẽ thực sự xảy ra nếu một lỗi xuất hiện trong 
\textbf{exp}?. Trong trường hợp đó, hầu như sẽ xảy ra tình huống
còn lại một lượng các token trong stack theo sau \textbf{stmts} 
mà chưa được reduce, và đồng thời cũng có những token 
sẽ cần phải đọc trước khi gặp được kí tự xuống dòng. Do vậy nên 
sản xuất phục hồi lỗi sẽ không được áp dụng theo một cách thông 
thường. 

Tuy nhiên Bison sẽ tìm cách sao cho trạng thái parser  phù hợp 
với sản xuất trên. Đầu tiên bằng cách loại bỏ các token và trạng 
thái trên stack sao cho nó trở về trạng thái mà token 
\textbf{error} có thể được chấp nhận. Sau đó, nó tiếp tục loại 
bỏ các token còn lại trong đầu vào cho đến khi gặp được kí tự 
phù hợp, trong trường hợp này là kí tự xuống dòng.

Với nhiều hợp, việc phục hồi lỗi không thể làm tốt hơn việc 
là đoán. Nếu mà parser đoán sai, một lỗi cú pháp có thể 
dẫn đến nhiều lỗi cú pháp khác. Để ngăn ngừa 
việc có quá nhiều thông báo lỗi cú pháp được in ra, parser 
sẽ không xuất ra thông báo lỗi cho những lỗi 
xuất hiện gần lỗi thứ nhất; chỉ khi sau 3 token liên tiếp 
từ đầu vào 
đã được chấp nhận bởi sản xuất khôi phục lỗi trước đó, 
thông báo lỗi mới có thể tiếp tục được in ra. 
\end{document}
