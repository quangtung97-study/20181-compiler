\documentclass[../report.tex]{subfiles}

\begin{document}
Mỗi file chỉ dẫn đầu vào cho Flex bao gồm 3 vùng, được phân tách nhau 
bởi các dòng chỉ chứa xâu '\%\%': \\
\lstinputlisting[style=customc]{src/input-file-format.l}

\noindent Trong đó: 
\begin{itemize}
\item \textbf{definitions}: Vùng chứa các định nghĩa. 
\item \textbf{rules}: Vùng chứa luật của các từ tố. 
\item \textbf{user code}: Vùng chứa mã C/C++ được thêm vào.
\end{itemize}

\subsection{Vùng định nghĩa - Definitions Section}
Vùng định nghĩa chứa các định nghĩa tên và các định nghĩa điều kiện bắt đầu.
Định nghĩa tên có dạng:
\begin{lstlisting}
name definition
\end{lstlisting}
\textbf{'name'} là các tên có cú pháp giống như các tên trong ngôn ngữ C.
\textbf{'definition'} là phần được lấy từ kí tự không phải kí tự trắng đầu tiên 
sau \textbf{'name'} và đến hết dòng. 
\textbf{'name'} không được thụt lề đầu dòng. 
Các định nghĩa có thể được tham chiếu đến bằng cách sử dụng \textbf{'\{name\}'} 
và nó sẽ được chuyển thành \textbf{'(definition)'}. \\
Ví dụ: 
\begin{lstlisting}
DIGIT [0-9]
\end{lstlisting}
Thì \textbf{'\{DIGIT\}'} sẽ được chuyển thành \textbf{'([0-9])'}.

Bất kì một dòng nào trong vùng này được viết thụt lề đầu dòng hoặc 
được bao xung quanh bởi \textbf{\%\{} và \textbf{\%\}} thì sẽ được sao chép nguyên văn 
vào file mã nguồn sinh ra. \\
Ví dụ: 
\lstinputlisting[style=customc]{src/unindented-line.l}
Thì đoạn code sau sẽ được sao chép nguyên văn vào file đầu ra: 
\lstinputlisting[style=customc]{src/unindented-line-output.c}

\subsection{Vùng các luật - Rules Section}
\subsection{Vùng mã nguồn bổ sung - User Code Section}

\end{document}
